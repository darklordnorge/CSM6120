AI agents deal with knowledge(data).
\begin{itemize}
\item Facts
\item Procedures
\item Meaning (relate \& define knowledge)
\end{itemize}

Right representation is crucial 
\begin{itemize}
\item Early realisation in AI
\item Wrong choice can lead to project failure
\item Active research area
\end{itemize}

\subsection{Choosing a representation}
For certain problem solving techniques.
\begin{itemize}
\item Best representation already known
\item Often a requirement of the technique
\item Or a requirement of the programming language(e.g. Prolog)
\end{itemize}

Representation of:
\begin{itemize}
\item Declarative knowledge(what, objects, structure)
\item Procedural knowledge(how, actions, performance)
\end{itemize}

\section{Aspects of Knowledge representation}
Syntax:
\begin{itemize}
\item Possible (allowed) constructions
\item For example: colour(my\_car, red), my\_car(red), red(my\_car), etc.
\end{itemize}

Semantics:
\begin{itemize}
\item What the representation \textbf{means} (and how it maps to the real world)
\item Example:
\begin{itemize}
\item Colour(my\_car, red) means: "my car is red", "paint my car red", etc.
\end{itemize}
\end{itemize}

Requirements for Knowledge Representation languages:\\
Representation adequacy:\\
\begin{itemize}
\item Should allow for representing all the required knowledge
\end{itemize}
Infernal adequacy:\\

Inferential efficiency:\\

Clear syntax and semantics:\\

Neutralness:\\

\section{What is a Logic}
A language with concrete rules.\\
No ambiguity in representation, however there may be errors. Allows for unambiguous communication and processing. \\
Is very unlike natural languages like e.g. English.\\

\subsection{Non-logical representation}
Logic representation have restrictions and can be hard to work with.

\subsection{What we've ignored}
Objects in the world tend to be related to each other.
\begin{itemize}
\item Classes, superclasses \& subclasses, part / whole hierarchies
\item Properties are \textit{inherited} across relationships
\end{itemize}

The state of the world can change over time.
\begin{itemize}
\item Explicit representation of time
\item Frame problem: representing the effects of action in logic without having to represent explicitly a large number of intuitive obvious non-effects
\item Non-monotonic reasoning
\end{itemize}

We must reason without complete knowledge
\begin{itemize}
\item Closed world assumption
\end{itemize}

Not all knowledge is "black \& white":
\begin{itemize}
\item Uncertainty, statistics, fuzzy logic,...
\end{itemize}

Defaults and exceptions:\\
Exception for a single object, a property of the object must be set to the (exception) value.\\

\section{Semantic Networks}
Semantic networks are essentially a generalization of inheritance hierarchies.\\
Each node is an object, class, concept, or event. \\

Each link is a relationship.
\begin{itemize}
\item is-a (the usual sublcass or element relationship
\item has-part or part-of
\item any other relationship that makes sense in context(e.g. owns thing x)
\end{itemize}

Semantic networks represent knowledge as a network or graph(easily stored on the computer). 
By traversing the network we can find:
\begin{itemize}
\item Elephant x likes apples(by inheritance)
\item That certain concepts related in certain ways(e.g. apples and elephants)
\end{itemize}

\section{Frames}
Devised by Marvin Minsky in 1974.\\
Is an extension to semantic networks, and incorporates certain valuable human thinking characteristics:\\
Expectations, assumptions, stereotypes, Exceptions, Fuzzy boundaries between classes.\\

Frames often allowed you to say which things were just typical(*) if a class, and which are definitional, so couldn't be overridden.\\
Frames also allow multiple inheritance(Nellie is an Elephant AND a circus animal).\\

\subsection{Frame representation}






